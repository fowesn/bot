\documentclass[a4paper, 12pt]{article}

\usepackage[T2A]{fontenc}
\usepackage[utf8]{inputenc}
\usepackage[russian]{babel}
\pagenumbering{gobble}
\usepackage{ragged2e}
\justifying 

\begin{document}
\noindent 

%%%%%    НАЧАЛО    %%%%%
%    Данный документ является шаблоном для файлов условий и алгоритмов решения заданий, используемых модулем наполнения системы контентом 
%    Текст условия или алгоритма решения необходимо вставить вместо данного блока комментариев
%    Для указания того, что на данное место необходимо подставить элемент входных данных, необходимо использовать спецификаторы PHP, например:
%        %d - для целых чисел,
%        %f - для чисел с плавающей точкой,
%        %s - для строк и т.д.
%    Для использования в качестве входных данных символов, являющихся специальными, необходимо их экранировать или использовать команды:
%        \% - символ процента
%        \$ - символ доллара
%        \& - амперсанд
%        \# - решетка
%        \_ - подчеркивание
%        \{ - левая фигурная скобка
%        \} - правая фигурная скобка
%        \backslash - обратная косая черта
%        \sim - тильда (математический режим)
%        \tilde - надстрочная тильда (математический режим)
%        \hat - крышка (математический режим)
%     Для включения математического режима необходимо окружить требуемый текст символами $ или \( и \)
%%%%%    КОНЕЦ    %%%%%

\end{document}